\pdfoutput=1
%% Author: PGL  Porta Mana
%% Created: 2015-05-01T20:53:34+0200
%% Last-Updated: 2022-04-09T21:46:42+0200
%%%%%%%%%%%%%%%%%%%%%%%%%%%%%%%%%%%%%%%%%%%%%%%%%%%%%%%%%%%%%%%%%%%%%%%%%%%%
\newif\ifarxiv
\arxivfalse
\iftrue\pdfmapfile{+classico.map}\fi
\newif\ifafour
\afourfalse% true = A4, false = A5
\newif\iftypodisclaim % typographical disclaim on the side
\typodisclaimtrue
\newcommand*{\memfontfamily}{zplx}
\newcommand*{\memfontpack}{newpxtext}
\documentclass[\ifafour a4paper,12pt,\else a5paper,10pt,\fi%extrafontsizes,%
onecolumn,oneside,article,%french,italian,german,swedish,latin,
british%
]{memoir}
\newcommand*{\firstdraft}{4 March 2022}
\newcommand*{\firstpublished}{\firstdraft}
\newcommand*{\updated}{\ifarxiv***\else\today\fi}
\newcommand*{\propertitle}{Guessing what's true or choosing what's optimal?\\{\Large A first-principle analysis of classifiers}%
}% title uses LARGE; set Large for smaller
\newcommand*{\pdftitle}{\propertitle}
\newcommand*{\headtitle}{Guessing truth or choosing optimality?}
\newcommand*{\pdfauthor}{K. Dirland, A. S. Lundervold, P.G.L.  Porta Mana}
\newcommand*{\headauthor}{Dirland, Lundervold, Porta Mana}
\newcommand*{\reporthead}{\ifarxiv\else Open Science Framework \href{https://doi.org/10.31219/osf.io/***}{\textsc{doi}:10.31219/osf.io/***}\fi}% Report number

%%%%%%%%%%%%%%%%%%%%%%%%%%%%%%%%%%%%%%%%%%%%%%%%%%%%%%%%%%%%%%%%%%%%%%%%%%%%
%%% Calls to packages (uncomment as needed)
%%%%%%%%%%%%%%%%%%%%%%%%%%%%%%%%%%%%%%%%%%%%%%%%%%%%%%%%%%%%%%%%%%%%%%%%%%%%

%\usepackage{pifont}

%\usepackage{fontawesome}

\usepackage[T1]{fontenc} 
\input{glyphtounicode} \pdfgentounicode=1

\usepackage[utf8]{inputenx}

%\usepackage{newunicodechar}
% \newunicodechar{Ĕ}{\u{E}}
% \newunicodechar{ĕ}{\u{e}}
% \newunicodechar{Ĭ}{\u{I}}
% \newunicodechar{ĭ}{\u{\i}}
% \newunicodechar{Ŏ}{\u{O}}
% \newunicodechar{ŏ}{\u{o}}
% \newunicodechar{Ŭ}{\u{U}}
% \newunicodechar{ŭ}{\u{u}}
% \newunicodechar{Ā}{\=A}
% \newunicodechar{ā}{\=a}
% \newunicodechar{Ē}{\=E}
% \newunicodechar{ē}{\=e}
% \newunicodechar{Ī}{\=I}
% \newunicodechar{ī}{\={\i}}
% \newunicodechar{Ō}{\=O}
% \newunicodechar{ō}{\=o}
% \newunicodechar{Ū}{\=U}
% \newunicodechar{ū}{\=u}
% \newunicodechar{Ȳ}{\=Y}
% \newunicodechar{ȳ}{\=y}

\newcommand*{\bmmax}{0} % reduce number of bold fonts, before font packages
\newcommand*{\hmmax}{0} % reduce number of heavy fonts, before font packages

\usepackage{textcomp}

%\usepackage[normalem]{ulem}% package for underlining
% \makeatletter
% \def\ssout{\bgroup \ULdepth=-.35ex%\UL@setULdepth
%  \markoverwith{\lower\ULdepth\hbox
%    {\kern-.03em\vbox{\hrule width.2em\kern1.2\p@\hrule}\kern-.03em}}%
%  \ULon}
% \makeatother

\usepackage{amsmath}

\usepackage{mathtools}
%\addtolength{\jot}{\jot} % increase spacing in multiline formulae
\setlength{\multlinegap}{0pt}

%\usepackage{empheq}% automatically calls amsmath and mathtools
%\newcommand*{\widefbox}[1]{\fbox{\hspace{1em}#1\hspace{1em}}}

%%%% empheq above seems more versatile than these:
%\usepackage{fancybox}
%\usepackage{framed}

% \usepackage[misc]{ifsym} % for dice
% \newcommand*{\diceone}{{\scriptsize\Cube{1}}}

\usepackage{amssymb}

\usepackage{amsxtra}

\usepackage[main=british]{babel}\selectlanguage{british}
%\newcommand*{\langnohyph}{\foreignlanguage{nohyphenation}}
\newcommand{\langnohyph}[1]{\begin{hyphenrules}{nohyphenation}#1\end{hyphenrules}}

\usepackage[autostyle=false,autopunct=false,english=british]{csquotes}
\setquotestyle{british}
\newcommand*{\defquote}[1]{`\,#1\,'}

% \makeatletter
% \renewenvironment{quotation}%
%                {\list{}{\listparindent 1.5em%
%                         \itemindent    \listparindent
%                         \rightmargin=1em   \leftmargin=1em
%                         \parsep        \z@ \@plus\p@}%
%                 \item[]\footnotesize}%
%                 {\endlist}
% \makeatother                


\usepackage{amsthm}
%% from https://tex.stackexchange.com/a/404680/97039
\makeatletter
\def\@endtheorem{\endtrivlist}
\makeatother

\newcommand*{\QED}{\textsc{q.e.d.}}
\renewcommand*{\qedsymbol}{\QED}
\theoremstyle{remark}
\newtheorem{note}{Note}
\newtheorem*{remark}{Note}
\newtheoremstyle{innote}{\parsep}{\parsep}{\footnotesize}{}{}{}{0pt}{}
\theoremstyle{innote}
\newtheorem*{innote}{}

\usepackage[shortlabels,inline]{enumitem}
\SetEnumitemKey{para}{itemindent=\parindent,leftmargin=0pt,listparindent=\parindent,parsep=0pt,itemsep=\topsep}
% \begin{asparaenum} = \begin{enumerate}[para]
% \begin{inparaenum} = \begin{enumerate*}
\setlist{itemsep=0pt,topsep=\parsep}
\setlist[enumerate,2]{label=\alph*.}
\setlist[enumerate]{label=\arabic*.,leftmargin=1.5\parindent}
\setlist[itemize]{leftmargin=1.5\parindent}
\setlist[description]{leftmargin=1.5\parindent}
% old alternative:
% \setlist[enumerate,2]{label=\alph*.}
% \setlist[enumerate]{leftmargin=\parindent}
% \setlist[itemize]{leftmargin=\parindent}
% \setlist[description]{leftmargin=\parindent}

\usepackage[babel,theoremfont,largesc]{newpxtext}

% For Baskerville see https://ctan.org/tex-archive/fonts/baskervillef?lang=en
% and http://mirrors.ctan.org/fonts/baskervillef/doc/baskervillef-doc.pdf
% \usepackage[p]{baskervillef}
% \usepackage[varqu,varl,var0]{inconsolata}
% \usepackage[scale=.95,type1]{cabin}
% \usepackage[baskerville,vvarbb]{newtxmath}
% \usepackage[cal=boondoxo]{mathalfa}


\usepackage[bigdelims,nosymbolsc%,smallerops % probably arXiv doesn't have it
]{newpxmath}
%\useosf
%\linespread{1.083}%
%\linespread{1.05}% widely used
\linespread{1.1}% best for text with maths
%% smaller operators for old version of newpxmath
\makeatletter
\def\re@DeclareMathSymbol#1#2#3#4{%
    \let#1=\undefined
    \DeclareMathSymbol{#1}{#2}{#3}{#4}}
%\re@DeclareMathSymbol{\bigsqcupop}{\mathop}{largesymbols}{"46}
%\re@DeclareMathSymbol{\bigodotop}{\mathop}{largesymbols}{"4A}
\re@DeclareMathSymbol{\bigoplusop}{\mathop}{largesymbols}{"4C}
\re@DeclareMathSymbol{\bigotimesop}{\mathop}{largesymbols}{"4E}
\re@DeclareMathSymbol{\sumop}{\mathop}{largesymbols}{"50}
\re@DeclareMathSymbol{\prodop}{\mathop}{largesymbols}{"51}
\re@DeclareMathSymbol{\bigcupop}{\mathop}{largesymbols}{"53}
\re@DeclareMathSymbol{\bigcapop}{\mathop}{largesymbols}{"54}
%\re@DeclareMathSymbol{\biguplusop}{\mathop}{largesymbols}{"55}
\re@DeclareMathSymbol{\bigwedgeop}{\mathop}{largesymbols}{"56}
\re@DeclareMathSymbol{\bigveeop}{\mathop}{largesymbols}{"57}
%\re@DeclareMathSymbol{\bigcupdotop}{\mathop}{largesymbols}{"DF}
%\re@DeclareMathSymbol{\bigcapplusop}{\mathop}{largesymbolsPXA}{"00}
%\re@DeclareMathSymbol{\bigsqcupplusop}{\mathop}{largesymbolsPXA}{"02}
%\re@DeclareMathSymbol{\bigsqcapplusop}{\mathop}{largesymbolsPXA}{"04}
%\re@DeclareMathSymbol{\bigsqcapop}{\mathop}{largesymbolsPXA}{"06}
\re@DeclareMathSymbol{\bigtimesop}{\mathop}{largesymbolsPXA}{"10}
%\re@DeclareMathSymbol{\coprodop}{\mathop}{largesymbols}{"60}
%\re@DeclareMathSymbol{\varprod}{\mathop}{largesymbolsPXA}{16}
\makeatother
%%
%% With euler font cursive for Greek letters - the [1] means 100% scaling
\DeclareFontFamily{U}{egreek}{\skewchar\font'177}%
\DeclareFontShape{U}{egreek}{m}{n}{<-6>s*[1]eurm5 <6-8>s*[1]eurm7 <8->s*[1]eurm10}{}%
\DeclareFontShape{U}{egreek}{m}{it}{<->s*[1]eurmo10}{}%
\DeclareFontShape{U}{egreek}{b}{n}{<-6>s*[1]eurb5 <6-8>s*[1]eurb7 <8->s*[1]eurb10}{}%
\DeclareFontShape{U}{egreek}{b}{it}{<->s*[1]eurbo10}{}%
\DeclareSymbolFont{egreeki}{U}{egreek}{m}{it}%
\SetSymbolFont{egreeki}{bold}{U}{egreek}{b}{it}% from the amsfonts package
\DeclareSymbolFont{egreekr}{U}{egreek}{m}{n}%
\SetSymbolFont{egreekr}{bold}{U}{egreek}{b}{n}% from the amsfonts package
% Take also \sum, \prod, \coprod symbols from Euler fonts
\DeclareFontFamily{U}{egreekx}{\skewchar\font'177}
\DeclareFontShape{U}{egreekx}{m}{n}{%
       <-7.5>s*[0.9]euex7%
    <7.5-8.5>s*[0.9]euex8%
    <8.5-9.5>s*[0.9]euex9%
    <9.5->s*[0.9]euex10%
}{}
\DeclareSymbolFont{egreekx}{U}{egreekx}{m}{n}
\DeclareMathSymbol{\sumop}{\mathop}{egreekx}{"50}
\DeclareMathSymbol{\prodop}{\mathop}{egreekx}{"51}
\DeclareMathSymbol{\coprodop}{\mathop}{egreekx}{"60}
\makeatletter
\def\sum{\DOTSI\sumop\slimits@}
\def\prod{\DOTSI\prodop\slimits@}
\def\coprod{\DOTSI\coprodop\slimits@}
\makeatother
\input{definegreek.tex}% Greek letters not usually given in LaTeX.

%\usepackage%[scaled=0.9]%
%{classico}%  Optima as sans-serif font
\renewcommand\sfdefault{uop}
\DeclareMathAlphabet{\mathsf}  {T1}{\sfdefault}{m}{sl}
\SetMathAlphabet{\mathsf}{bold}{T1}{\sfdefault}{b}{sl}
%\newcommand*{\mathte}[1]{\textbf{\textit{\textsf{#1}}}}
% Upright sans-serif math alphabet
% \DeclareMathAlphabet{\mathsu}  {T1}{\sfdefault}{m}{n}
% \SetMathAlphabet{\mathsu}{bold}{T1}{\sfdefault}{b}{n}

% DejaVu Mono as typewriter text
\usepackage[scaled=0.84]{DejaVuSansMono}

\usepackage{mathdots}

\usepackage[usenames]{xcolor}
% Tol (2012) colour-blind-, print-, screen-friendly colours, alternative scheme; Munsell terminology
\definecolor{mypurpleblue}{RGB}{68,119,170}
\definecolor{myblue}{RGB}{102,204,238}
\definecolor{mygreen}{RGB}{34,136,51}
\definecolor{myyellow}{RGB}{204,187,68}
\definecolor{myred}{RGB}{238,102,119}
\definecolor{myredpurple}{RGB}{170,51,119}
\definecolor{mygrey}{RGB}{187,187,187}
% Tol (2012) colour-blind-, print-, screen-friendly colours; Munsell terminology
% \definecolor{lbpurple}{RGB}{51,34,136}
% \definecolor{lblue}{RGB}{136,204,238}
% \definecolor{lbgreen}{RGB}{68,170,153}
% \definecolor{lgreen}{RGB}{17,119,51}
% \definecolor{lgyellow}{RGB}{153,153,51}
% \definecolor{lyellow}{RGB}{221,204,119}
% \definecolor{lred}{RGB}{204,102,119}
% \definecolor{lpred}{RGB}{136,34,85}
% \definecolor{lrpurple}{RGB}{170,68,153}
\definecolor{lgrey}{RGB}{221,221,221}
%\newcommand*\mycolourbox[1]{%
%\colorbox{mygrey}{\hspace{1em}#1\hspace{1em}}}
\colorlet{shadecolor}{lgrey}

\usepackage{bm}

\usepackage{microtype}

\usepackage[backend=biber,mcite,%subentry,
citestyle=authoryear-comp,bibstyle=pglpm-authoryear,autopunct=false,sorting=ny,sortcites=false,natbib=false,maxcitenames=2,maxbibnames=8,minbibnames=8,giveninits=true,uniquename=false,uniquelist=false,maxalphanames=1,block=space,hyperref=true,defernumbers=false,useprefix=true,sortupper=false,language=british,parentracker=false]{biblatex}
\DeclareSortingTemplate{ny}{\sort{\field{sortname}\field{author}\field{editor}}\sort{\field{year}}}
\iffalse\makeatletter%%% replace parenthesis with brackets
\newrobustcmd*{\parentexttrack}[1]{%
  \begingroup
  \blx@blxinit
  \blx@setsfcodes
  \blx@bibopenparen#1\blx@bibcloseparen
  \endgroup}
\AtEveryCite{%
  \let\parentext=\parentexttrack%
  \let\bibopenparen=\bibopenbracket%
  \let\bibcloseparen=\bibclosebracket}
\makeatother\fi
\DefineBibliographyExtras{british}{\def\finalandcomma{\addcomma}}
\renewcommand*{\finalnamedelim}{\addspace\amp\space}
% \renewcommand*{\finalnamedelim}{\addcomma\space}
\renewcommand*{\textcitedelim}{\addcomma\space}
% \setcounter{biburlnumpenalty}{1} % to allow url breaks anywhere
% \setcounter{biburlucpenalty}{0}
% \setcounter{biburllcpenalty}{1}
\DeclareDelimFormat{multicitedelim}{\addsemicolon\addspace\space}
\DeclareDelimFormat{compcitedelim}{\addsemicolon\addspace\space}
\DeclareDelimFormat{postnotedelim}{\addspace}
\ifarxiv\else\addbibresource{portamanabib.bib}\fi
\renewcommand{\bibfont}{\footnotesize}
%\appto{\citesetup}{\footnotesize}% smaller font for citations
\defbibheading{bibliography}[\bibname]{\section*{#1}\addcontentsline{toc}{section}{#1}%\markboth{#1}{#1}
}
\newcommand*{\citep}{\footcites}
\newcommand*{\citey}{\footcites}%{\parencites*}
\newcommand*{\ibid}{\unspace\addtocounter{footnote}{-1}\footnotemark{}}
%\renewcommand*{\cite}{\parencite}
%\renewcommand*{\cites}{\parencites}
\providecommand{\href}[2]{#2}
\providecommand{\eprint}[2]{\texttt{\href{#1}{#2}}}
\newcommand*{\amp}{\&}
% \newcommand*{\citein}[2][]{\textnormal{\textcite[#1]{#2}}%\addtocategory{extras}{#2}
% }
\newcommand*{\citein}[2][]{\textnormal{\textcite[#1]{#2}}%\addtocategory{extras}{#2}
}
\newcommand*{\citebi}[2][]{\textcite[#1]{#2}%\addtocategory{extras}{#2}
}
\newcommand*{\subtitleproc}[1]{}
\newcommand*{\chapb}{ch.}
%
%\def\UrlOrds{\do\*\do\-\do\~\do\'\do\"\do\-}%
\def\myUrlOrds{\do\0\do\1\do\2\do\3\do\4\do\5\do\6\do\7\do\8\do\9\do\a\do\b\do\c\do\d\do\e\do\f\do\g\do\h\do\i\do\j\do\k\do\l\do\m\do\n\do\o\do\p\do\q\do\r\do\s\do\t\do\u\do\v\do\w\do\x\do\y\do\z\do\A\do\B\do\C\do\D\do\E\do\F\do\G\do\H\do\I\do\J\do\K\do\L\do\M\do\N\do\O\do\P\do\Q\do\R\do\S\do\T\do\U\do\V\do\W\do\X\do\Y\do\Z}%
\makeatletter
%\g@addto@macro\UrlSpecials{\do={\newline}}
\g@addto@macro{\UrlBreaks}{\myUrlOrds}
\makeatother
\newcommand*{\arxiveprint}[1]{%
arXiv \doi{10.48550/arXiv.#1}%
}
\newcommand*{\mparceprint}[1]{%
\href{http://www.ma.utexas.edu/mp_arc-bin/mpa?yn=#1}{mp\_arc:\allowbreak\nolinkurl{#1}}%
}
\newcommand*{\haleprint}[1]{%
\href{https://hal.archives-ouvertes.fr/#1}{\textsc{hal}:\allowbreak\nolinkurl{#1}}%
}
\newcommand*{\philscieprint}[1]{%
\href{http://philsci-archive.pitt.edu/archive/#1}{PhilSci:\allowbreak\nolinkurl{#1}}%
}
\newcommand*{\doi}[1]{%
\href{https://doi.org/#1}{\textsc{doi}:\allowbreak\nolinkurl{#1}}%
}
\newcommand*{\biorxiveprint}[1]{%
bioRxiv \doi{10.1101/#1}%
}
\newcommand*{\osfeprint}[1]{%
Open Science Framework \doi{10.31219/osf.io/#1}%
}

\usepackage{graphicx}

%\usepackage{wrapfig}

%\usepackage{tikz-cd}

\PassOptionsToPackage{hyphens}{url}\usepackage[hypertexnames=false,pdfencoding=unicode,psdextra]{hyperref}

\usepackage[depth=4]{bookmark}
\hypersetup{colorlinks=true,bookmarksnumbered,pdfborder={0 0 0.25},citebordercolor={0.2667 0.4667 0.6667},citecolor=mypurpleblue,linkbordercolor={0.6667 0.2 0.4667},linkcolor=myredpurple,urlbordercolor={0.1333 0.5333 0.2},urlcolor=mygreen,breaklinks=true,pdftitle={\pdftitle},pdfauthor={\pdfauthor}}
% \usepackage[vertfit=local]{breakurl}% only for arXiv
\providecommand*{\urlalt}{\href}

\usepackage[british]{datetime2}
\DTMnewdatestyle{mydate}%
{% definitions
\renewcommand*{\DTMdisplaydate}[4]{%
\number##3\ \DTMenglishmonthname{##2} ##1}%
\renewcommand*{\DTMDisplaydate}{\DTMdisplaydate}%
}
\DTMsetdatestyle{mydate}

%%%%%%%%%%%%%%%%%%%%%%%%%%%%%%%%%%%%%%%%%%%%%%%%%%%%%%%%%%%%%%%%%%%%%%%%%%%%
%%% Layout. I do not know on which kind of paper the reader will print the
%%% paper on (A4? letter? one-sided? double-sided?). So I choose A5, which
%%% provides a good layout for reading on screen and save paper if printed
%%% two pages per sheet. Average length line is 66 characters and page
%%% numbers are centred.
%%%%%%%%%%%%%%%%%%%%%%%%%%%%%%%%%%%%%%%%%%%%%%%%%%%%%%%%%%%%%%%%%%%%%%%%%%%%
\ifafour\setstocksize{297mm}{210mm}%{*}% A4
\else\setstocksize{210mm}{5.5in}%{*}% 210x139.7
\fi
\settrimmedsize{\stockheight}{\stockwidth}{*}
\setlxvchars[\normalfont] %313.3632pt for a 66-characters line
\setxlvchars[\normalfont]
% \setlength{\trimtop}{0pt}
% \setlength{\trimedge}{\stockwidth}
% \addtolength{\trimedge}{-\paperwidth}
%\settrims{0pt}{0pt}
% The length of the normalsize alphabet is 133.05988pt - 10 pt = 26.1408pc
% The length of the normalsize alphabet is 159.6719pt - 12pt = 30.3586pc
% Bringhurst gives 32pc as boundary optimal with 69 ch per line
% The length of the normalsize alphabet is 191.60612pt - 14pt = 35.8634pc
\ifafour\settypeblocksize{*}{32pc}{1.618} % A4
%\setulmargins{*}{*}{1.667}%gives 5/3 margins % 2 or 1.667
\else\settypeblocksize{*}{26pc}{1.618}% nearer to a 66-line newpx and preserves GR
\fi
\setulmargins{*}{*}{1}%gives equal margins
\setlrmargins{*}{*}{*}
\setheadfoot{\onelineskip}{2.5\onelineskip}
\setheaderspaces{*}{2\onelineskip}{*}
\setmarginnotes{2ex}{10mm}{0pt}
\checkandfixthelayout[nearest]
%%% End layout
%% this fixes missing white spaces
%\pdfmapline{+dummy-space <dummy-space.pfb}
%\pdfinterwordspaceon% seems to add a white margin to Sumatrapdf

%%% Sectioning
\newcommand*{\asudedication}[1]{%
{\par\centering\textit{#1}\par}}
\newenvironment{acknowledgements}{\section*{Thanks}\addcontentsline{toc}{section}{Thanks}}{\par}
\makeatletter\renewcommand{\appendix}{\par
  \bigskip{\centering
   \interlinepenalty \@M
   \normalfont
   \printchaptertitle{\sffamily\appendixpagename}\par}
  \setcounter{section}{0}%
  \gdef\@chapapp{\appendixname}%
  \gdef\thesection{\@Alph\c@section}%
  \anappendixtrue}\makeatother
\counterwithout{section}{chapter}
\setsecnumformat{\upshape\csname the#1\endcsname\quad}
\setsecheadstyle{\large\bfseries\sffamily%
\centering}
\setsubsecheadstyle{\bfseries\sffamily%
\raggedright}
%\setbeforesecskip{-1.5ex plus 1ex minus .2ex}% plus 1ex minus .2ex}
%\setaftersecskip{1.3ex plus .2ex }% plus 1ex minus .2ex}
%\setsubsubsecheadstyle{\bfseries\sffamily\slshape\raggedright}
%\setbeforesubsecskip{1.25ex plus 1ex minus .2ex }% plus 1ex minus .2ex}
%\setaftersubsecskip{-1em}%{-0.5ex plus .2ex}% plus 1ex minus .2ex}
\setsubsecindent{0pt}%0ex plus 1ex minus .2ex}
\setparaheadstyle{\bfseries\sffamily%
\raggedright}
\setcounter{secnumdepth}{2}
\setlength{\headwidth}{\textwidth}
\newcommand{\addchap}[1]{\chapter*[#1]{#1}\addcontentsline{toc}{chapter}{#1}}
\newcommand{\addsec}[1]{\section*{#1}\addcontentsline{toc}{section}{#1}}
\newcommand{\addsubsec}[1]{\subsection*{#1}\addcontentsline{toc}{subsection}{#1}}
\newcommand{\addpara}[1]{\paragraph*{#1.}\addcontentsline{toc}{subsubsection}{#1}}
\newcommand{\addparap}[1]{\paragraph*{#1}\addcontentsline{toc}{subsubsection}{#1}}

%%% Headers, footers, pagestyle
\copypagestyle{manaart}{plain}
\makeheadrule{manaart}{\headwidth}{0.5\normalrulethickness}
\makeoddhead{manaart}{%
{\footnotesize%\sffamily%
\scshape\headauthor}}{}{{\footnotesize\sffamily%
\headtitle}}
\makeoddfoot{manaart}{}{\thepage}{}
\newcommand*\autanet{\includegraphics[height=\heightof{M}]{autanet.pdf}}
\definecolor{mygray}{gray}{0.333}
\iftypodisclaim%
\ifafour\newcommand\addprintnote{\begin{picture}(0,0)%
\put(245,149){\makebox(0,0){\rotatebox{90}{\tiny\color{mygray}\textsf{This
            document is designed for screen reading and
            two-up printing on A4 or Letter paper}}}}%
\end{picture}}% A4
\else\newcommand\addprintnote{\begin{picture}(0,0)%
\put(176,112){\makebox(0,0){\rotatebox{90}{\tiny\color{mygray}\textsf{This
            document is designed for screen reading and
            two-up printing on A4 or Letter paper}}}}%
\end{picture}}\fi%afourtrue
\makeoddfoot{plain}{}{\makebox[0pt]{\thepage}\addprintnote}{}
\else
\makeoddfoot{plain}{}{\makebox[0pt]{\thepage}}{}
\fi%typodisclaimtrue
\makeoddhead{plain}{\scriptsize\reporthead}{}{}
% \copypagestyle{manainitial}{plain}
% \makeheadrule{manainitial}{\headwidth}{0.5\normalrulethickness}
% \makeoddhead{manainitial}{%
% \footnotesize\sffamily%
% \scshape\headauthor}{}{\footnotesize\sffamily%
% \headtitle}
% \makeoddfoot{manaart}{}{\thepage}{}

\pagestyle{manaart}

\setlength{\droptitle}{-3.9\onelineskip}
\pretitle{\begin{center}\LARGE\sffamily%
\bfseries}
\posttitle{\bigskip\end{center}}

\makeatletter\newcommand*{\atf}{\includegraphics[totalheight=\heightof{@}]{atblack.png}}\makeatother
\providecommand{\affiliation}[1]{\textsl{\textsf{\footnotesize #1}}}
\providecommand{\epost}[1]{\texttt{\footnotesize\textless#1\textgreater}}
\providecommand{\email}[2]{\href{mailto:#1ZZ@#2 ((remove ZZ))}{#1\protect\atf#2}}
%\providecommand{\email}[2]{\href{mailto:#1@#2}{#1@#2}}

\preauthor{\vspace{-0.5\baselineskip}\begin{center}
\normalsize\sffamily%
\lineskip  0.5em}
\postauthor{\par\end{center}}
\predate{\DTMsetdatestyle{mydate}\begin{center}\footnotesize}
\postdate{\end{center}\vspace{-\medskipamount}}

\setfloatadjustment{figure}{\footnotesize}
\captiondelim{\quad}
\captionnamefont{\footnotesize\sffamily%
}
\captiontitlefont{\footnotesize}
%\firmlists*
\midsloppy
% handling orphan/widow lines, memman.pdf
% \clubpenalty=10000
% \widowpenalty=10000
% \raggedbottom
% Downes, memman.pdf
\clubpenalty=9996
\widowpenalty=9999
\brokenpenalty=4991
\predisplaypenalty=10000
\postdisplaypenalty=1549
\displaywidowpenalty=1602
\raggedbottom

\paragraphfootnotes
\setlength{\footmarkwidth}{2ex}
% \threecolumnfootnotes
%\setlength{\footmarksep}{0em}
\footmarkstyle{\textsuperscript{%\color{myred}
\scriptsize\bfseries#1}~}
%\footmarkstyle{\textsuperscript{\color{myred}\scriptsize\bfseries#1}~}
%\footmarkstyle{\textsuperscript{[#1]}~}

\selectlanguage{british}\frenchspacing

%%%%%%%%%%%%%%%%%%%%%%%%%%%%%%%%%%%%%%%%%%%%%%%%%%%%%%%%%%%%%%%%%%%%%%%%%%%%
%%% Paper's details
%%%%%%%%%%%%%%%%%%%%%%%%%%%%%%%%%%%%%%%%%%%%%%%%%%%%%%%%%%%%%%%%%%%%%%%%%%%%
\title{\propertitle}
\author{%
\hspace*{\stretch{1}}%
%% uncomment if additional authors present
\parbox{0.3\linewidth}%\makebox[0pt][c]%
{\protect\centering K. Dirland\\%
\footnotesize\epost{\email{***}{***}}}%
\hspace*{\stretch{1}}%
\parbox{0.3\linewidth}%\makebox[0pt][c]%
{\protect\centering A. S. Lundervold\\%
\footnotesize\epost{\email{***}{***}}}%
\hspace*{\stretch{1}}%
\parbox{0.3\linewidth}%\makebox[0pt][c]%
{\protect\centering P.G.L.  Porta Mana  \href{https://orcid.org/0000-0002-6070-0784}{\protect\includegraphics[scale=0.16]{orcid_32x32.png}}\\\footnotesize\epost{\email{pgl}{portamana.org}}}%
% Mohn Medical Imaging and Visualization Centre, Dept of Computer science, Electrical Engineering and Mathematical Sciences, Western Norway University of Applied Sciences, Bergen, Norway
%% uncomment if additional authors present
% \hspace*{\stretch{1}}%
% \parbox{0.5\linewidth}%\makebox[0pt][c]%
% {\protect\centering ***\\%
% \footnotesize\epost{\email{***}{***}}}%
\hspace*{\stretch{1}}%
\\\scriptsize(or any permutation thereof)
}

%\date{Draft of \today\ (first drafted \firstdraft)}
\date{\textbf{Draft}. \firstpublished; updated \updated}

%%%%%%%%%%%%%%%%%%%%%%%%%%%%%%%%%%%%%%%%%%%%%%%%%%%%%%%%%%%%%%%%%%%%%%%%%%%%
%%% Macros @@@
%%%%%%%%%%%%%%%%%%%%%%%%%%%%%%%%%%%%%%%%%%%%%%%%%%%%%%%%%%%%%%%%%%%%%%%%%%%%

% Common ones - uncomment as needed
%\providecommand{\nequiv}{\not\equiv}
%\providecommand{\coloneqq}{\mathrel{\mathop:}=}
%\providecommand{\eqqcolon}{=\mathrel{\mathop:}}
%\providecommand{\varprod}{\prod}
\newcommand*{\de}{\partialup}%partial diff
\newcommand*{\pu}{\piup}%constant pi
\newcommand*{\delt}{\deltaup}%Kronecker, Dirac
%\newcommand*{\eps}{\varepsilonup}%Levi-Civita, Heaviside
%\newcommand*{\riem}{\zetaup}%Riemann zeta
%\providecommand{\degree}{\textdegree}% degree
%\newcommand*{\celsius}{\textcelsius}% degree Celsius
%\newcommand*{\micro}{\textmu}% degree Celsius
\newcommand*{\I}{\mathrm{i}}%imaginary unit
\newcommand*{\e}{\mathrm{e}}%Neper
\newcommand*{\di}{\mathrm{d}}%differential
%\newcommand*{\Di}{\mathrm{D}}%capital differential
%\newcommand*{\planckc}{\hslash}
%\newcommand*{\avogn}{N_{\textrm{A}}}
%\newcommand*{\NN}{\bm{\mathrm{N}}}
%\newcommand*{\ZZ}{\bm{\mathrm{Z}}}
%\newcommand*{\QQ}{\bm{\mathrm{Q}}}
\newcommand*{\RR}{\bm{\mathrm{R}}}
%\newcommand*{\CC}{\bm{\mathrm{C}}}
%\newcommand*{\nabl}{\bm{\nabla}}%nabla
%\DeclareMathOperator{\lb}{lb}%base 2 log
%\DeclareMathOperator{\tr}{tr}%trace
%\DeclareMathOperator{\card}{card}%cardinality
%\DeclareMathOperator{\im}{Im}%im part
%\DeclareMathOperator{\re}{Re}%re part
%\DeclareMathOperator{\sgn}{sgn}%signum
%\DeclareMathOperator{\ent}{ent}%integer less or equal to
%\DeclareMathOperator{\Ord}{O}%same order as
%\DeclareMathOperator{\ord}{o}%lower order than
%\newcommand*{\incr}{\triangle}%finite increment
\newcommand*{\defd}{\coloneqq}
\newcommand*{\defs}{\eqqcolon}
%\newcommand*{\Land}{\bigwedge}
%\newcommand*{\Lor}{\bigvee}
%\newcommand*{\lland}{\DOTSB\;\land\;}
%\newcommand*{\llor}{\DOTSB\;\lor\;}
\newcommand*{\limplies}{\mathbin{\Rightarrow}}%implies
%\newcommand*{\suchthat}{\mid}%{\mathpunct{|}}%such that (eg in sets)
%\newcommand*{\with}{\colon}%with (list of indices)
%\newcommand*{\mul}{\times}%multiplication
%\newcommand*{\inn}{\cdot}%inner product
%\newcommand*{\dotv}{\mathord{\,\cdot\,}}%variable place
%\newcommand*{\comp}{\circ}%composition of functions
%\newcommand*{\con}{\mathbin{:}}%scal prod of tensors
%\newcommand*{\equi}{\sim}%equivalent to 
\renewcommand*{\asymp}{\simeq}%equivalent to 
%\newcommand*{\corr}{\mathrel{\hat{=}}}%corresponds to
%\providecommand{\varparallel}{\ensuremath{\mathbin{/\mkern-7mu/}}}%parallel (tentative symbol)
\renewcommand*{\le}{\leqslant}%less or equal
\renewcommand*{\ge}{\geqslant}%greater or equal
%\DeclarePairedDelimiter\clcl{[}{]}
%\DeclarePairedDelimiter\clop{[}{[}
%\DeclarePairedDelimiter\opcl{]}{]}
%\DeclarePairedDelimiter\opop{]}{[}
\DeclarePairedDelimiter\abs{\lvert}{\rvert}
%\DeclarePairedDelimiter\norm{\lVert}{\rVert}
\DeclarePairedDelimiter\set{\{}{\}} %}
%\DeclareMathOperator{\pr}{P}%probability
\newcommand*{\p}{\mathrm{p}}%probability
\renewcommand*{\P}{\mathrm{P}}%probability
\newcommand*{\E}{\mathrm{E}}
%% The "\:" space is chosen to correctly separate inner binary and external rels
\renewcommand*{\|}[1][]{\nonscript\:#1\vert\nonscript\:\mathopen{}}
%\DeclarePairedDelimiterX{\cp}[2]{(}{)}{#1\nonscript\:\delimsize\vert\nonscript\:\mathopen{}#2}
%\DeclarePairedDelimiterX{\ct}[2]{[}{]}{#1\nonscript\;\delimsize\vert\nonscript\:\mathopen{}#2}
%\DeclarePairedDelimiterX{\cs}[2]{\{}{\}}{#1\nonscript\:\delimsize\vert\nonscript\:\mathopen{}#2}
%\newcommand*{\+}{\lor}
%\renewcommand{\*}{\land}
%% symbol = for equality statements within probabilities
%% from https://tex.stackexchange.com/a/484142/97039
% \newcommand*{\eq}{\mathrel{\!=\!}}
% \let\texteq\=
% \renewcommand*{\=}{\TextOrMath\texteq\eq}
% \newcommand*{\eq}[1][=]{\mathrel{\!#1\!}}
\newcommand*{\mo}[1][=]{\mathrel{\mkern-3.5mu#1\mkern-3.5mu}}
%\newcommand*{\moo}[1][=]{\mathrel{\!#1\!}}
%\newcommand*{\mo}[1][=]{\mathord{#1}}
%\newcommand*{\mo}[1][=]{\mathord{\,#1\,}}
%%
\newcommand*{\sect}{\S}% Sect.~
\newcommand*{\sects}{\S\S}% Sect.~
\newcommand*{\chap}{ch.}%
\newcommand*{\chaps}{chs}%
\newcommand*{\bref}{ref.}%
\newcommand*{\brefs}{refs}%
%\newcommand*{\fn}{fn}%
\newcommand*{\eqn}{eq.}%
\newcommand*{\eqns}{eqs}%
\newcommand*{\fig}{fig.}%
\newcommand*{\figs}{figs}%
\newcommand*{\vs}{{vs}}
\newcommand*{\eg}{{e.g.}}
\newcommand*{\etc}{{etc.}}
\newcommand*{\ie}{{i.e.}}
%\newcommand*{\ca}{{c.}}
\newcommand*{\foll}{{ff.}}
%\newcommand*{\viz}{{viz}}
\newcommand*{\cf}{{cf.}}
%\newcommand*{\Cf}{{Cf.}}
%\newcommand*{\vd}{{v.}}
\newcommand*{\etal}{{et al.}}
%\newcommand*{\etsim}{{et sim.}}
%\newcommand*{\ibid}{{ibid.}}
%\newcommand*{\sic}{{sic}}
%\newcommand*{\id}{\mathte{I}}%id matrix
%\newcommand*{\nbd}{\nobreakdash}%
%\newcommand*{\bd}{\hspace{0pt}}%
%\def\hy{-\penalty0\hskip0pt\relax}
%\newcommand*{\labelbis}[1]{\tag*{(\ref{#1})$_\text{r}$}}
%\newcommand*{\mathbox}[2][.8]{\parbox[t]{#1\columnwidth}{#2}}
%\newcommand*{\zerob}[1]{\makebox[0pt][l]{#1}}
\newcommand*{\tprod}{\mathop{\textstyle\prod}\nolimits}
\newcommand*{\tsum}{\mathop{\textstyle\sum}\nolimits}
%\newcommand*{\tint}{\begingroup\textstyle\int\endgroup\nolimits}
%\newcommand*{\tland}{\mathop{\textstyle\bigwedge}\nolimits}
%\newcommand*{\tlor}{\mathop{\textstyle\bigvee}\nolimits}
%\newcommand*{\sprod}{\mathop{\textstyle\prod}}
%\newcommand*{\ssum}{\mathop{\textstyle\sum}}
%\newcommand*{\sint}{\begingroup\textstyle\int\endgroup}
%\newcommand*{\sland}{\mathop{\textstyle\bigwedge}}
%\newcommand*{\slor}{\mathop{\textstyle\bigvee}}
%\newcommand*{\T}{^\transp}%transpose
%%\newcommand*{\QEM}%{\textnormal{$\Box$}}%{\ding{167}}
%\newcommand*{\qem}{\leavevmode\unskip\penalty9999 \hbox{}\nobreak\hfill
%\quad\hbox{\QEM}}

%%%%%%%%%%%%%%%%%%%%%%%%%%%%%%%%%%%%%%%%%%%%%%%%%%%%%%%%%%%%%%%%%%%%%%%%%%%%
%%% Custom macros for this file @@@
%%%%%%%%%%%%%%%%%%%%%%%%%%%%%%%%%%%%%%%%%%%%%%%%%%%%%%%%%%%%%%%%%%%%%%%%%%%%
\definecolor{notecolour}{RGB}{68,170,153}
%\newcommand*{\puzzle}{\maltese}
\newcommand*{\puzzle}{{\fontencoding{U}\fontfamily{fontawesometwo}\selectfont\symbol{225}}}
\newcommand*{\wrench}{{\fontencoding{U}\fontfamily{fontawesomethree}\selectfont\symbol{114}}}
\newcommand*{\pencil}{{\fontencoding{U}\fontfamily{fontawesometwo}\selectfont\symbol{210}}}
\newcommand{\mynote}[1]{ {\color{notecolour}#1}}

\newcommand*{\widebar}[1]{{\mkern1.5mu\skew{2}\overline{\mkern-1.5mu#1\mkern-1.5mu}\mkern 1.5mu}}

% \newcommand{\explanation}[4][t]{%\setlength{\tabcolsep}{-1ex}
% %\smash{
% \begin{tabular}[#1]{c}#2\\[0.5\jot]\rule{1pt}{#3}\\#4\end{tabular}}%}
% \newcommand*{\ptext}[1]{\text{\small #1}}
%\DeclareMathOperator*{\argsup}{arg\,sup}
\newcommand*{\dob}{degree of belief}
\newcommand*{\dobs}{degrees of belief}
\newcommand*{\ml}{machine-learning}
\newcommand*{\Fs}{F_{\textrm{s}}}
\newcommand*{\fs}{f_{\textrm{s}}}
\newcommand*{\uF}{\bar{F}}
\newcommand*{\uf}{\bar{f}}
\newcommand*{\za}{\hat{0}}
\newcommand*{\zb}{\hat{1}}
\newcommand*{\eu}{\bar{U}}

%%% Custom macros end @@@

%%%%%%%%%%%%%%%%%%%%%%%%%%%%%%%%%%%%%%%%%%%%%%%%%%%%%%%%%%%%%%%%%%%%%%%%%%%%
%%% Beginning of document
%%%%%%%%%%%%%%%%%%%%%%%%%%%%%%%%%%%%%%%%%%%%%%%%%%%%%%%%%%%%%%%%%%%%%%%%%%%%
%\firmlists
\begin{document}
\captiondelim{\quad}\captionnamefont{\footnotesize}\captiontitlefont{\footnotesize}
\selectlanguage{british}\frenchspacing
\maketitle

%%%%%%%%%%%%%%%%%%%%%%%%%%%%%%%%%%%%%%%%%%%%%%%%%%%%%%%%%%%%%%%%%%%%%%%%%%%%
%%% Abstract
%%%%%%%%%%%%%%%%%%%%%%%%%%%%%%%%%%%%%%%%%%%%%%%%%%%%%%%%%%%%%%%%%%%%%%%%%%%%
\abstractrunin
\abslabeldelim{}
\renewcommand*{\abstractname}{}
\setlength{\absleftindent}{0pt}
\setlength{\absrightindent}{0pt}
\setlength{\abstitleskip}{-\absparindent}
\begin{abstract}\labelsep 0pt%
  \noindent ***abstract***
% \\\noindent\emph{\footnotesize Note: Dear Reader
%     \amp\ Peer, this manuscript is being peer-reviewed by you. Thank you.}
% \par%\\[\jot]
% \noindent
% {\footnotesize PACS: ***}\qquad%
% {\footnotesize MSC: ***}%
%\qquad{\footnotesize Keywords: ***}
\end{abstract}
\selectlanguage{british}\frenchspacing

%%%%%%%%%%%%%%%%%%%%%%%%%%%%%%%%%%%%%%%%%%%%%%%%%%%%%%%%%%%%%%%%%%%%%%%%%%%%
%%% Epigraph
%%%%%%%%%%%%%%%%%%%%%%%%%%%%%%%%%%%%%%%%%%%%%%%%%%%%%%%%%%%%%%%%%%%%%%%%%%%%
% \asudedication{\small ***}
% \vspace{\bigskipamount}
% \setlength{\epigraphwidth}{.7\columnwidth}
% %\epigraphposition{flushright}
% \epigraphtextposition{flushright}
% %\epigraphsourceposition{flushright}
% \epigraphfontsize{\footnotesize}
% \setlength{\epigraphrule}{0pt}
% %\setlength{\beforeepigraphskip}{0pt}
% %\setlength{\afterepigraphskip}{0pt}
% \epigraph{\emph{text}}{source}



%%%%%%%%%%%%%%%%%%%%%%%%%%%%%%%%%%%%%%%%%%%%%%%%%%%%%%%%%%%%%%%%%%%%%%%%%%%%
%%% BEGINNING OF MAIN TEXT
%%%%%%%%%%%%%%%%%%%%%%%%%%%%%%%%%%%%%%%%%%%%%%%%%%%%%%%%%%%%%%%%%%%%%%%%%%%%

\mynote{\scriptsize\puzzle\ [Luca] I find it very difficult to structure the paper: there seems to be issues at several levels in the development and use of binary classifiers (and classifiers in general) within machine-learning.

  Here are some relevant points:
  \begin{itemize}
  \item There should be a distinction between \enquote{inference} (or forecast, prediction, guess) and \enquote{decision} (or action, choice). In particular, the possible situations we may be uncertain about and the possible decisions available may be completely different things. A clinician, for example, may be uncertain about \enquote{cancer} vs \enquote{non-cancer}, while the choices are about \enquote{drug treatment 1} vs \enquote{drug treatment 2} vs \enquote{surgery}.

  \item Probability theory \amp\ decision theory say that in order to make self-consistent decision we need two things: (a) the probabilities for the possible situations, (b) the utilities of the decisions given each possible situation.

  \item A useful \ml\ algorithm should therefore give us one of two things:
\begin{itemize}
\item either the \emph{probabilities} of the uncertain situations
  (\enquote{cancer} vs \enquote{non-cancer} in the example above),
\item or the final decision (\enquote{drug treatment 1} vs \enquote{drug
    treatment 2} vs \enquote{surgery} in the example above).
\end{itemize}
Current \ml\ classifiers do not give us either: the output in the example
above would be \enquote{cancer} vs \enquote{non-cancer}, often without
probabilities.

\item So there are two possible solutions to the problem above:
  \begin{itemize}
  \item We must build a classifier that outputs probabilities. The 0--1
    outputs of current classifiers cannot properly interpreted as
    probabilities, for various reasons.
  \item We must build a classifier that output \emph{decisions}: so not
    \enquote{cancer} vs \enquote{non-cancer}, but \enquote{drug treatment
      1} vs \etc.
  \end{itemize}

\end{itemize}
}




\section{Valuation metrics, amounts of data, inferences, and decisions}
\label{sec:intro}

Let's consider the simple example of a binary classifier and several dilemmas that appear in its development, choice, and use.

At the moment of evaluating different classifier algorithms, or different
hyperparameter settings for one algorithm, we are avalanched by a choice of
possible evaluation scales: accuracy, area under curve, $F_{1}$-measure,
mean square contingency \citep[denoted \enquote{$r$} there]{yule1912} also
known as Matthews correlation coefficient \citep{matthews1975}, precision,
recall, sensitivity, specificity, and many others
\citep{sammutetal2011_r2017}. Only vague guidelines are usually given to
face this choice. A thorough analysis and discussion of several such scales
was given by
\textcite{goodmanetal1954,goodmanetal1959,goodmanetal1963,goodmanetal1972b}.

We can also ask: are all these scales well-founded and self-consistent? is
it possible that the use of any of them leads to contradictions? The
literature abounds with studies showing that some scale $X$ may imply
hidden contradictions with the data or the assumptions used for our
inference, and is therefore worse than some other scale $Y$. See for
example \textcite{bakeretal2001,loboetal2008,handetal2018,zhu2020} and
Goodman \amp\ Kruskal's papers cited above for instances of criticisms of
area under the curve, $F_{1}$-measure, Matthews correlation coefficient,
and other scales.

If we have many more data for one class than for the other -- a common
predicament in medical applications -- we must face the
\enquote{class-imbalance problem}: the classifier ends up classifying all
data as belonging to the more numerous class
\citep{sammutetal2011_r2017,provost2000}, which may be an undesirable
action if the misclassification of cases belonging to the less numerous
class entails high costs.

The three points above turn out to be tightly related and to have a common
solution. We show that
\begin{enumerate}
\item\label{item:2D_family} the admissible valuation scales for a binary
  classifier form a two-dimensional family; that is, the choice of a specific
  scale corresponds to the choice of two numbers. Such choice is
  problem-dependent and cannot be given a priori.
\item\label{item:inconsistent} admissible scales are only those that can be
  expressed as a linear function of the elements of the
  population-normalized confusion matrix. Scales such as the
  $F_{1}$-measure or the Matthews correlation coefficient are therefore inad***
\end{enumerate}




\section{Overview of decision theory}
\label{sec:decision_theory}

 \citep[\chap~15]{russelletal1995_r2022}{jeffrey1965}{north1968}

Decision theory makes a distinction between
\begin{enumerate}[label=\alph*.]
\item the possible situations we are uncertain about
\item the possible choices we can make.
\end{enumerate}
This distinction is important, in fact in some cases the numbers of possible uncertain situations




% \[ \color{mypurpleblue}\bm{a} \color{myredpurple}\mathbin{\bm{\land}}\color{mypurpleblue} \bm{b} \]


%%%% examples use empheq
%   \begin{empheq}[left={\mathllap{\begin{aligned}    \de\yF_{\yc}/\de\yp&=0\text{:} \\
%         \de\yF_{\yc}/\de\ym&=0\text{:}\\ \de\yF_{\yc}/\de\yl&=0\text{:}\end{aligned}}\qquad}\empheqlbrace]{align}
%     \label{eq:con_p}
% %    \de\yF_{\yc}/\de\yp &\equiv
%     -\ln\yp + \ln\yq + \yl\yM + \ym\yu &=0,\\
%     \label{eq:con_u}
% %    \de\yF_{\yc}/\de\ym &\equiv
%     \yu\yp-1 &=0,\\
%     \label{eq:con_l}
%     %\de\yF_{\yc}/\de\yl &\equiv
%     \yM\yp-\yc &=0.
%   \end{empheq}
%%%%
% \begin{empheq}[box=\widefbox]{equation}
%   \label{eq:maxent_question}
%   \p\bigl[\yE{N+1}{k} \bigcond \tsum\yo\yf{N}\in\yA, \yM\bigr] = \mathord{?}
% \end{empheq}


%%\setlength{\intextsep}{0ex}% with wrapfigure
%%\setlength{\columnsep}{0ex}% with wrapfigure
%\begin{figure}[p!]% with figure
%\begin{wrapfigure}{r}{0.4\linewidth} % with wrapfigure
%  \centering\includegraphics[trim={12ex 0 18ex 0},clip,width=\linewidth]{maxent_saddle.png}\\
%\caption{caption}\label{fig:comparison_a5}
%\end{figure}% exp_family_maxent.nb


%%%%%%%%%%%%%%%%%%%%%%%%%%%%%%%%%%%%%%%%%%%%%%%%%%%%%%%%%%%%%%%%%%%%%%%%%%%%
%%% Acknowledgements
%%%%%%%%%%%%%%%%%%%%%%%%%%%%%%%%%%%%%%%%%%%%%%%%%%%%%%%%%%%%%%%%%%%%%%%%%%%% 
\iffalse
\begin{acknowledgements}
  \ldots to Mari \amp\ Miri for continuous encouragement and affection, and
  to Buster Keaton and Saitama for filling life with awe and inspiration.
  To the developers and maintainers of \LaTeX, Emacs, AUC\TeX, Open Science
  Framework, R, Python, Inkscape, Sci-Hub for making a free and impartial
  scientific exchange possible.
  % Our work was supported by the Trond Mohn Research Foundation, grant number BFS2018TMT07
%\rotatebox{15}{P}\rotatebox{5}{I}\rotatebox{-10}{P}\rotatebox{10}{\reflectbox{P}}\rotatebox{-5}{O}.
%\sourceatright{\autanet}
\mbox{}\hfill\autanet
\end{acknowledgements}
\fi

%%%%%%%%%%%%%%%%%%%%%%%%%%%%%%%%%%%%%%%%%%%%%%%%%%%%%%%%%%%%%%%%%%%%%%%%%%%%
%%% Appendices
%%%%%%%%%%%%%%%%%%%%%%%%%%%%%%%%%%%%%%%%%%%%%%%%%%%%%%%%%%%%%%%%%%%%%%%%%%%% 
%\clearpage
\bigskip
\renewcommand*{\appendixpagename}{}
% \renewcommand*{\appendixname}{Appendix: test2}
% %\appendixpage
\appendix

\addsec{Appendix: broader overview of binary classification}
% \label{sec:test}

Let us consider our binary-classification problem from a general perspective and summarize how it would be approached and solved from first principles\citep[part~IV]{russelletal1995_r2022} if our computational resources had no constraints.

In our long-term task we will receive \enquote{units} of a specific kind; the units for example could be gadgets, individuals, or investment portfolios. Each new unit will belong to one of two classes, which we can denote $X\mo 0$ and $X\mo 1$; for example they could be \enquote{defective} vs \enquote{non-defective}, \enquote{ill} vs \enquote{healthy}. The class will be unknown to us. For each new unit we shall need to decide among two possible actions, which we can denote $A\mo\za$ and $A\mo\zb$; for example \enquote{discard} vs \enquote{keep}, or \enquote{treat} vs \enquote{dismiss}. The utility of each action depends on the unknown class of the unit; we denote these utilities by $U(A \| X)$. For each new unit we will be able to measure a \enquote{feature} $Z$ of a specific kind common to all units; for example $Z$ could be a set of categorical and real quantities, or an image such as a brain scan. We have a set of units -- our \enquote{sample units} or \enquote{sample data} -- that are somehow \enquote*{representative} of the units we will receive in our long-term task \citep[for a critical analysis of the sometimes hollow term \enquote{representative sample} see][]{kruskaletal1979,kruskaletal1979b,kruskaletal1979c,kruskaletal1980}. we know both the class and the feature of each of these sample units. Let us denote this sample information by $D$.

According to the principles of decision theory and probability theory, for each new unit we would proceed as follows:
\begin{enumerate}[label=\arabic*.]
\item Assign probabilities to the two possible values of the unit's class, given the value of the unit's feature $Z\mo z$, our sample data $D$, and any other available information:
  \begin{equation}
    \p(X\mo 0 \| Z\mo z, D), \qquad \p(X\mo 1 \| Z\mo z,D) \equiv 1- \p(X\mo 0 \| Z\mo z,D) \ ,
  \end{equation}
  according to the rules of the probability calculus.
\item Calculate the expected utilities $\eu$ of the two possible actions:
  \begin{equation}
    \begin{aligned}
      \eu(\za) &\defd U(\za \| X\mo 0) \ 
                 \p(X\mo 0 \| Z\mo z, D) + U(\za \| X\mo 1) \ 
                 \p(X\mo 1 \| Z\mo z, D)
      \\
      \eu(\zb) &\defd U(\zb \| X\mo 0) \ 
                 \p(X\mo 0 \| Z\mo z, D) + U(\zb \| X\mo 1) \ 
                 \p(X\mo 1 \| Z\mo z, D)
    \end{aligned}
\end{equation}
  and choose the action having maximal expected utility.
\end{enumerate}

\medskip

How is the probability $\p(X \| Z\mo z, D)$ determined by the probability calculus? Here is a simplified, intuitive picture. First consider the case where the feature $Z$ can only assume a small number of possible values, so that many units can in principle have the same value of $Z$.

Consider the collection of all units having $Z\mo z$ that we received in the past and will receive in the future. Among them, a proportion $F(X\mo 0 \| Z\mo z)$ belong to class $0$, and a proportion $1 - F(X\mo 0 \| Z\mo z) \equiv F(X\mo 1 \| Z\mo z)$ to class $1$. For example these two proportions could be 74\% and 26\%. Our present unit with $Z\mo z$ is a member of this collection. The probability $\p(X\mo 0 \| Z\mo z)$ that our unit belongs to class $0$, given that its feature has value $z$, is then intuitively equal to the proportion $F(X\mo 0 \| Z\mo z)$. Analogously for $X\mo 1$.

The problem is that we do not know the proportion $F(X\mo 0 \| Z\mo z)$. However, we expect it to be roughly equal to the analogous proportion seen in our sample data; let us denote the latter by $\Fs(X\mo 0 \| Z\mo z)$:
\begin{equation}
  \label{eq:approx_repres}
  F(X\mo 0 \| Z\mo z) \sim \Fs(X\mo 0 \| Z\mo z) \ .
\end{equation}
this is indeed what we mean by saying that our sample data are \enquote{representative} of the future units. Later we shall discuss the case in which such representativeness is of different kinds. We expect the discrepancy between $F(X\mo 0 \| Z\mo z)$ and $\Fs(X\mo 0 \| Z\mo z)$ to be smaller, the larger the number of sample data. Vice versa we expect it to be larger, the smaller the number of sample data.

If $Z$ can assume a continuum of values, as is the case for a brain scan for example, then the collection of units having $Z\mo z$ is more difficult to imagine. In this case each unit will be unique in its feature value -- no two brains are exactly alike.




\mynote{\medskip\hrule old text below}

Given the unit's feature $Z$ we will assign probabilities to the possible values of the unit's class:  according to the rules of the probability calculus.

As mentioned in \sect~\ref{sec:decision_theory}, a decision problem under uncertainty is conceptually divided into two steps 

The Suppose we have a population of units or individuals characterized by a possibly multidimensional variable $Z$ and a binary variable $X \in \set{0,1}$. Different joint combinations of $(X,Z)$ values can appear in this population. Denote by $F(X\mo x, Z\mo z)$, or more simply $F(x, z)$ when there is no confusion, the number of individuals having specific joint values $(X\mo x, Z\mo z)$. This is the absolute frequency of the values $(x,z)$. We can also count the number of individuals having a specific value of $Z\mo z$, regardless of $X$; this is the marginal absolute frequency $F(z)$. It is easy to see that
\begin{equation}
  \label{eq:marginal_prob}
  F(z) = F(X\mo 0, z) + F(X\mo 1, z) \equiv \sum_{x} F(x,z)\ .
\end{equation}
Analogously for $F(x)$.

Select only the subpopulation of individuals that have a specific value $Z\mo z$. In this subpopulation, the \emph{proportion} of individuals having a specific value $X\mo x$ is $f(x\| Z\mo z)$. This is the conditional relative frequency of $x$ given that $z$. It is easy to see that
\begin{equation}
  \label{eq:cond_prob}
  f(x \| z) = \frac{F(x,z)}{F(z)} \ .
\end{equation}

Now suppose that we know all these statistics about this population. An
individual coming from this population is presented to us. We measure its
$Z$ and obtain the value $z$. What could be the value of $X$ for this
individual? We know that among all individuals having $Z\mo z$ (and the
individual before us is one of them) a proportion $f(x \| z)$ has $X\mo x$.
Thus we can say that there is a probability $f(x \| z)$ that our individual
has $X\mo x$. And this is all we can say if we only know $Z$.

\medskip

For this individual we must choose among two actions $\set{a, b}$. The
utility of performing action $a$ if the individual has $X\mo x$, and given
any other known circumstances, is $U(a \| x)$; similarly for $b$. If we
knew the value of $X$, say $X\mo 0$, we would simply choose the action
leading to maximal utility:
\begin{equation}
  \label{eq:choice_ex}
  \begin{aligned}
    &\text{if}\quad U(a \| X\mo 0) > U(b \| X\mo 0) \quad\text{then choose action $a$},
\\
      &\text{if}\quad U(a \| X\mo 0) < U(b \| X\mo 0) \quad\text{then choose action $b$},
\\&\text{else}\quad\text{it does not matter which action is chosen}.
  \end{aligned}
\end{equation}
But we do not know the actual value of $X$. We have probabilities for the
possible values of $X$ given that $Z\mo z$ for our individual. Since $X$ is
uncertain, the final utilities of the two actions are also uncertain; but we can
calculate their \emph{expected} values $\bar{U}(a \| Z \mo z)$ and
$\bar{U}(b \| Z \mo z)$:
\begin{equation}
  \label{eq:expe_util}
  \begin{aligned}
    &\bar{U}(a \| z) \defd
    U(a \| X\mo 0)\ f(X\mo 0 \| z) + U(a \| X\mo 1)\ f(X\mo 1 \| z) \ ,
    \\
    &\bar{U}(b \| z) \defd
    U(b \| X\mo 0)\ f(X\mo 0 \| z) + U(b \| X\mo 1)\ f(X\mo 1 \| z) \ .
\end{aligned}
\end{equation}
Decision theory shows that the optimal action is the one having the maximal
expected utility. Our choice therefore proceeds as follows:
\begin{equation}
  \label{eq:choice_uncertain}
  \begin{aligned}
    &\text{if}\quad \bar{U}(a \| z) > \bar{U}(b \| z) \quad\text{then choose action $a$},
\\
      &\text{if}\quad \bar{U}(a \| z) < \bar{U}(b \| z) \quad\text{then choose action $b$},
\\&\text{else}\quad\text{it does not matter which action is chosen}.
  \end{aligned}
\end{equation}

\medskip

The decision procedure just discussed is very simple and does not need any machine-learning algorithms. It could be implemented in a simple algorithm that takes as input the full statistics $F(X,Z)$ of the population, the utilities, and yields an output according to~\eqref{eq:choice_uncertain}.

Our main problem is that the full statistics $F(X,Z)$ is almost universally not known. Typically we only have the statistics $\Fs(X,Z)$ of a sample of individuals that come from the population of interest or from populations that are somewhat related to the one of interest. This is where probability theory steps in. It allows us to assign probabilities to all the possible statistics $F(X,Z)$. From these probabilities we can calculate the \emph{expected} value $\uf(x \| z)$ of the conditional frequencies $f(x \| z)$. Decision theory says that the expected value $\uf(x \| z)$ should then be used, in this uncertain case, in \eqn~\eqref{eq:expe_util} in place of the unknown $f(x \| z)$. The decision procedure~\eqref{eq:choice_uncertain} can then be used again.

Probability theory says that in this particular situation the probability of a particular possible statistics $F(X,Z)$ is the product of two factors having intuitive interpretations:
\begin{itemize}
\item the probability of observing the statistics $\Fs(X,Z)$ of our data sample, assuming the full statistics to be $F(X,Z)$. With some combinatorics it can be shown that this probability is proportional to
  \begin{equation}
    \label{eq:likelihood_relentropy}
%    \exp\biggl[\sum_{X,Z}\Fs(X,Z) \ln \frac{\Fs(X,Z)}{F(X,Z)}\biggr]
    \exp\biggl[\sum_{X,Z}\Fs(X,Z) \ln F(X,Z)\biggr]
  \end{equation}
  The argument of the exponential is the cross-entropy between $\Fs(X,Z)$ and $F(X,Z)$; this is the reason of its appearance in the loss function used for classifiers \citep{bridle1990,mackay1992d}.

  This factor tells us how much the possible statistics \emph{fit} the sample data; it gives more weight to statistics with a better fit.
  
\item the probability of the full statistics $F(X,Z)$ for reasons not present in the data, for example because of physical laws, biological plausibility, or similar.

This factor tells us whether the possible statistics should be favourably considered, or maybe even discarded instead, for reasons that go beyond the data we have seen; in other words, whether the hypothetical statistics would \emph{generalize} well beyond the sample data.  
\end{itemize}
The final probability comes from the balance between these \enquote{fit} and \enquote{generalization} factors. Note that the first factor becomes more important as the sample size and therefore $\Fs(X,Z)$ increases; the sample data eventually determine what the most probable statistics is, if the sample is large enough.

A similar probabilistic reasoning applies if our sample data come not from
the population of interest but from a population having at least the same
\emph{conditional} frequencies of as the one of interest, either
$f(X \| Z)$ or $f(Z \| X)$. The latter case must be examined with care when
our purpose is to guess $X$ from $Z$. In this case we cannot use the
conditional frequencies $\fs(X \| Z)$ that appear in the data to obtain the
expected value $\uf(X \| Z)$: they could be completely different from the
ones of the population of interest. We must instead use the sample
conditional frequencies $\fs(Z \| X)$ to obtain the expected value
$\uf(Z \| X)$, and then combine the latter with an appropriate probability
$P(X)$ through Bayes's theorem:
\begin{equation}
  \label{eq:bayes_app}
  \frac{\uf(Z \| X)\ P(X)}{\sum_{X} \uf(Z \| X)\ P(X)} \ .
\end{equation}
The probability $P(X)$ cannot be obtained from the data, but requires a separate study or survey. In medical applications, where $X$ represents for example the presence or absence of a disease, the probability $P(X)$ is the base rate of the disease. Direct use of $\fs(X \| Z)$ from the data instead of \eqref{eq:bayes_app} is the \enquote{base-rate fallacy} \citep[\sect~12.5]{russelletal1995_r2022}{axelsson2000,jennyetal2018}.




In supervised learning the classifier is trained to learn the most probable $f(X \| Z)$ from the data. The training finds the $f(X \| Z)$ that most closely fits the conditional frequency $\fs(X \| Z)$ of the sampled data; this roughly corresponds to maximizing the first factor \eqref{eq:likelihood_relentropy} described above. The architecture and the parameter regularizer of the classifier play the role of the second factor.



%%%%%%%%%%%%%%%%%%%%%%%%%%%%%%%%%%%%%%%%%%%%%%%%%%%%%%%%%%%%%%%%%%%%%%%%%%%%
%%% Bibliography
%%%%%%%%%%%%%%%%%%%%%%%%%%%%%%%%%%%%%%%%%%%%%%%%%%%%%%%%%%%%%%%%%%%%%%%%%%%% 
\renewcommand*{\finalnamedelim}{\addcomma\space}
\defbibnote{prenote}{{\footnotesize (\enquote{de $X$} is listed under D,
    \enquote{van $X$} under V, and so on, regardless of national
    conventions.)\par}}
% \defbibnote{postnote}{\par\medskip\noindent{\footnotesize% Note:
%     \arxivp \mparcp \philscip \biorxivp}}

\printbibliography[prenote=prenote%,postnote=postnote
]

\end{document}

%%%%%%%%%%%%%%%%%%%%%%%%%%%%%%%%%%%%%%%%%%%%%%%%%%%%%%%%%%%%%%%%%%%%%%%%%%%%
%%% Cut text (won't be compiled)
%%%%%%%%%%%%%%%%%%%%%%%%%%%%%%%%%%%%%%%%%%%%%%%%%%%%%%%%%%%%%%%%%%%%%%%%%%%% 


%%% Local Variables: 
%%% mode: LaTeX
%%% TeX-PDF-mode: t
%%% TeX-master: t
%%% End: 
